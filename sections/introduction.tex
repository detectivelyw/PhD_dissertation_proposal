% this is an introduction and a problem statement

% TODO The introduction should hint on what I am planning to expose on this 
% I guess I have to make it patent that I want to present how this PhD is going
% to go for the rest of the program

% FIXME: this seems to focus on introducing the supply chain, but not my work
% and stuff. I definitely need to rewrite this
\paragraph{Introduction}\mbox{}\\

A {\it software supply chain} is the series of steps performed when writing,
testing, packaging, and distributing software.  A typical software supply chain
is composed of multiple steps ``chained'' together to transform (e.g.,
compilation) or verify the state (e.g., linting) of the project in order to
drive it into a {\it final product}. Usually, a software supply chain starts
with the inclusion of code to the version control system and ends with the
software's installation on an user's system.

Supply chain security is crucial to the overall security of a software product.
An attacker who is able to control a step in the supply chain can modify the
product for malicious intents that range from introducing backdoors in the
source code to including vulnerable libraries in the final product. Because it
is susceptible to these threats, supply chain breaches are an impactful means
for an attacker to affect multiple users at once. For instance, Juniper, a
widely deployed firewall supplier, was infiltrated by an attacker who
introduced a backdoor that made millions of devices vulnerable for three
years~\cite{juniper-backdoor}. Examples of this are not lacking, as many major
companies have been compromised in the same way, causing millions of dollars in
damages~\cite{aurora-scm, adobe-hack, github-hacked-millions}.

Currently, supply chain security is addressed only by securing each individual
step within it. For example, git commit signing~\cite{git-sign} can be used to
verify that only allowed parties introduce code into the repository. This can
successfully deter some attacks on source code repositories. Integrity and
security measures, which provide similar assurance to git commit signing, are
also available for many steps in the supply chain~\cite{reprobuilds, codeship,
twistlock, clair-coreos, signed-push}.

Unfortunately, step-specific security measures are not sufficient because there
are no mechanisms that verify the integrity of the supply chain as a whole ---
or even ensure these security measures were performed in each individual
step. Although the security of each individual step is critical, attackers can
modify the output of a step before it is fed to the next and subvert any
security measures~\cite{juniper-backdoor, aurora-scm, linux-backdoor-2003}. As is oftentimes
demonstrated, the security of each individual link in the chain does not
necessarily result in a secure supply chain~\cite{redhat-hacked,
linux-mint-hack, xcode}. Evidence of this is widespread: in the case of the Juniper hack, a compromise in their source
code repository allowed attackers to modify the random-number
generator configuration used in certain firewall models. Though this was a trivial compromise, it allowed
attackers to eavesdrop on every single VPN conversation that used those Juniper
firewall models~\cite{juniper-backdoor}.

In order to address these issues, we created {\it in-toto} (Latin for ``as a
whole,''). in-toto is the first system that holistically enforces the integrity
of a software supply chain by gathering information about the it and
passing this information forward to the end user for verification. This
information about the supply chain's layout and the results of each individual
step is gathered in the form of {\it supply chain layout} (i.e., what steps
need to be done) and {\it link metadata} (i.e., information about the result of
each individual step), and can be used by the client to make sure appropriate
supply chain security practices were performed on each step.
