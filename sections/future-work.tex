\paragraph{Ongoing work and future goals}\mbox{}\\

Throughout the remainder of my PhD studies I intend to explore various topics
that are consequential of in-toto's integration. The Software Supply Chain is a
recent, thriving field, full of areas to explore and, thanks to in-toto and its
integrations, a trove of data to investigate. in-toto will allow us to work on
a myriad of academically-relevant projects.

% TODO: mention how we are going to work on other aspects of the supply chain 
First, in-toto's cryptographically signed and tamper-resistant log of all
changes makes it possible to gain new insights on all the actions that affect
and change a software product. By gathering and learning from in-toto metadata,
it will be possible not only to judge and block insecure supply chain layouts,
but to study and improve existing supply chain practices. We envision work in
this field that will cover aspects such as finding correlations between such
practices and the presence of vulnerabilities (e.g., how often is the lack of
a code review associated with stack overflows?). As a satellite effect of this
work, we aim to increase the visibility of a project's supply chain practices
to foster their improvement through user and community pressure and thus
raise the bar for secure development practices in the industry.

% TODO talk a little bit about PKI
Second, key management and distribution for in-toto is an open problem. While
there are point solutions to this issue that successfully apply to other
systems (e.g., ACME~\cite{acme} or CONIKS~\cite{coniks}), we foresee problems
(like project namespacing) that may require us to design our own solution.
Aspects of key management include implicit and explicit revocation of final
products as they are found vulnerable (i.e., it is possible that a step used in
many final products is found vulnerable and could be automatically revoked).
Other examples of this nature are layouts that yield multiple final products,
as well as layouts that are used as a part of other layouts (these we call
sublayouts).

% TODO: here I mention stuff like functionary in a box
Finally, working towards ensuring secure environments for the execution of each
step is paramount. In this regard, we envision functionary on a box, a
security-hardened host with safe defaults and reasonable interfaces to perform
steps in a supply chain with increased assurance. Other areas, such as data
provenance, would be helpful to provide a traceable execution path that can be
further analyzed.

% TODO: I need a closing statement
The current state of affairs on supply chain security leaves a lot of questions
unanswered and --- even worse --- questions unasked. I am hopeful that the
remainder of this PhD studies will lay a foundation for a topic that has been
overlooked for a long time, yet is in dire need of specialization. I am hopeful
that, by the end of my studies, we can see in-toto as a widely-used standard for
software supply chain security.
