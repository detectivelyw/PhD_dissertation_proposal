\paragraph{Goals and initial progress}\mbox{}\\

% if we are going to talk about the goals, i should really have a paragraph of
% so explaining what we are trying to achieve.
The main goal of this research is to create, design, implement and ensure the
adoption of in-toto. I intend to work towards this goal through three main
conceptual axes: 1) analyzing and securing each individual link within the
supply chain to provide baseline security guarantees, 2) implementing of a
production-ready version of in-toto, and 3) integrating and contributing with
widely-deployed products (such as Docker).

% TODO: here, i need to talk about the git-attacks paper, and the git patches
As a first take on this project, we've studied the security properties of
popular tools that are typically used in software supply chains. Namely, we've looked at the
security of version control systems (VCS), as well as initiatives such as
reproducible builds~\cite{reprobuilds}. The goal with this initial approach was to ensure the
security of individual supply chain elements, for we can't ensure the supply
chain as a whole if the security of individual links is lacking. It is well
known that a chain is only as strong as its weakest link.

These initial analyses were quite fruitful. As a result of these first
approaches we were able to identify several vulnerabilities (read, design
flaws), in the way the Git~\cite{git} VCS handles the information served to
users when synchronizing remote repositories. This work culminated in not only
a USENIX paper~\cite{torres2016omitting}, but three series of 13 patches in total to fix the issues.
Users of Git starting from version 2.13~\cite{git-commits-santiago, git-commits-lukas, git-213-contributors} already benefit from our
fixes. This work is an ongoing effort, as this series of patches do not fix all
the underlying issues we've identified. We will continue to work with Git throughout the upcoming year to complete the integration of our fixes.

As another outcome of this research, we identified issues with the status quo
regarding security practices in VCSs, which culminated in the first
Javascript-only implementation of Git merge and a native-browser,
highly-optimized remote Git signing algorithm. This work is currently under
submission to ACM CCS 2017~\cite{ACMCCS2017}. 
% TODO: the sentence below is crap, we need to seay something more
%We expect work on this nature to advance
%the state-of-the art of security in the first element on the supply chain.

% FIXME: I don't like this crap.
%At the time of this writing, we leave the version-control-systems with stronger
%security assurances and move on to further steps in the supply chain, such as
%the build and testing stages. We have approached initiatives such as
%reproducible builds and explored ways in which we can integrate them as part of
%a secure supply chain. Reproducible builds present themselves as 

In parallel to the focused work on individual steps in the supply chain, we
have been able to take initial steps to securing the supply chain as a whole.
On this front, we have published version 1.0 of the in-toto specification~\cite{in-toto-spec}, as
well as an accompanying reference implementation~\cite{in-toto-impl}. As a complement, we have
presented and introduced in-toto to several open source communities, including
Docker~\cite{docker}, Debian~\cite{debian}, Arch Linux~\cite{archlinux}, and
OpenSUSE~\cite{opensuse}, among others. These approaches have been received
positively after the presentation at Dockercon 2017~\cite{dockercon2017, dockercon2017talk}
(April 17) and the ensuing invitation to DebConf~\cite{debconf2017}.

% FIXME: this is shite
%While we succeded in these first steps, there is still substantial work to be
%done. I believe that, in order to succeed in my PhD studies, I need to advance
%the state of the art in software supply chain security. This involves
%designing, constructing a robust system, as well as guaranteeing its adoption.
%I aim to make in-toto the de-facto standard of software supply chain security.

% this is the in-depth description of the system.
% FIXME: \lois{Maybe a sentence or so of the
% CONSEQUENCES to a product and future users if the chain is corrupted.}


%\paragraph{short-term goals}\mbox{}\\

The release of the in-toto specification and reference implementations, as well
as the outreach to communities hints that the immediate goals are to have
popular and widely used products (such as Debian, Arch Linux and Docker) 
integrate in-toto in production. While there is clear interest from these and
other communities, the effort to integrate these systems is non-trivial, and
requires a lot of collaboration between multiple parties.  As we approach
communities, we identify unforeseen needs and requirements that are specific to a
project. In this sense, we expect to drift from the original specification to
accommodate these needs. From the community's feedback, we will produce an
updated specification with production grade integrations by the third quarter
of this year.


